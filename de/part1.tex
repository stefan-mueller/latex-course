\documentclass{beamer}

\input{preamble.tex}

\subtitle{Part 1: Grundlagen \\ 
(\href{https://github.com/jdleesmiller/latex-course}{basierend auf den Folien von Dr. John D. Lees-Miller})}

\begin{document}

%%%%%%%%%%%%%%%%%%%%%%%%%%%%%%%%%%%%%%%%%%%%%%%%%%%%%%%%%%%%%%%%%%%%%%%%%%%%%%%
%%%%%%%%%%%%%%%%%%%%%%%%%%%%%%%%%%%%%%%%%%%%%%%%%%%%%%%%%%%%%%%%%%%%%%%%%%%%%%%
%%%%%%%%%%%%%%%%%%%%%%%%%%%%%%%%%%%%%%%%%%%%%%%%%%%%%%%%%%%%%%%%%%%%%%%%%%%%%%%
\begin{frame}
\titlepage
\end{frame}

%%%%%%%%%%%%%%%%%%%%%%%%%%%%%%%%%%%%%%%%%%%%%%%%%%%%%%%%%%%%%%%%%%%%%%%%%%%%%%%
%%%%%%%%%%%%%%%%%%%%%%%%%%%%%%%%%%%%%%%%%%%%%%%%%%%%%%%%%%%%%%%%%%%%%%%%%%%%%%%
%%%%%%%%%%%%%%%%%%%%%%%%%%%%%%%%%%%%%%%%%%%%%%%%%%%%%%%%%%%%%%%%%%%%%%%%%%%%%%%
\begin{frame}{Warum \LaTeX{}?}
\begin{itemize}
\item Es erstellt wunderschöne Dokumente
\begin{itemize}
\item Insbesondere mathematische Formeln
\end{itemize}
%
\item Erstellt von Wissenschaftlern für Wissenschaftler 
\begin{itemize}
\item Eine große und aktive Community
\end{itemize}
%
\item Es ist leistungsfähig -- und Du kannst es erweiteren
\begin{itemize}
\item Packages für Papers, Presentations, Poster, Lebensläufe, \ldots
\end{itemize}
\end{itemize}
\end{frame}

%%%%%%%%%%%%%%%%%%%%%%%%%%%%%%%%%%%%%%%%%%%%%%%%%%%%%%%%%%%%%%%%%%%%%%%%%%%%%%%
%%%%%%%%%%%%%%%%%%%%%%%%%%%%%%%%%%%%%%%%%%%%%%%%%%%%%%%%%%%%%%%%%%%%%%%%%%%%%%%
%%%%%%%%%%%%%%%%%%%%%%%%%%%%%%%%%%%%%%%%%%%%%%%%%%%%%%%%%%%%%%%%%%%%%%%%%%%%%%%
\begin{frame}[fragile]{Wie funktioniert es?}
\begin{itemize}
\item Schreibe das Dokument in \texttt{plain text} mit \cmd{Befehlen}, die 
die dern Struktur und Bedeutung definieren. 
\item Das \texttt{latex} Programm verarbeitetet den Text und die Befehle
und erstellt wundervoll formatierte Dokumente.
\end{itemize}
\vskip 2ex
\begin{center}
\begin{minted}[frame=single]{latex}
\LaTeX ist ein wirklich \emph{hilfreiches} Programm.
\end{minted}
\vskip 2ex
\tikz\node[single arrow,fill=gray,font=\ttfamily\bfseries,%
  rotate=270,xshift=-1em]{latex};
\vskip 2ex
\fbox{\LaTeX ist ein wirklich \emph{hilfreiches} Programm.}
\end{center}
\end{frame}

%%%%%%%%%%%%%%%%%%%%%%%%%%%%%%%%%%%%%%%%%%%%%%%%%%%%%%%%%%%%%%%%%%%%%%%%%%%%%%%
%%%%%%%%%%%%%%%%%%%%%%%%%%%%%%%%%%%%%%%%%%%%%%%%%%%%%%%%%%%%%%%%%%%%%%%%%%%%%%%
%%%%%%%%%%%%%%%%%%%%%%%%%%%%%%%%%%%%%%%%%%%%%%%%%%%%%%%%%%%%%%%%%%%%%%%%%%%%%%%
\begin{frame}[fragile]{Weitere Beispiele für Befehle und deren Ergebnis\ldots}
\begin{exampletwoup}
\begin{itemize}
\item Tee
\item Milch
\item Kekse
\end{itemize}
\end{exampletwoup}
\vskip 2ex
\begin{exampletwoup}
\begin{figure}
\includegraphics{kuecken}
\end{figure}
\end{exampletwoup}
\vskip 2ex
\begin{exampletwoup}
\begin{equation}
\alpha + \beta + 1
\end{equation}
\end{exampletwoup}

\tiny{Bild basierend auf \url{http://www.andy-roberts.net/writing/latex/importing_images}.}
\end{frame}

%%%%%%%%%%%%%%%%%%%%%%%%%%%%%%%%%%%%%%%%%%%%%%%%%%%%%%%%%%%%%%%%%%%%%%%%%%%%%%%
%%%%%%%%%%%%%%%%%%%%%%%%%%%%%%%%%%%%%%%%%%%%%%%%%%%%%%%%%%%%%%%%%%%%%%%%%%%%%%%
%%%%%%%%%%%%%%%%%%%%%%%%%%%%%%%%%%%%%%%%%%%%%%%%%%%%%%%%%%%%%%%%%%%%%%%%%%%%%%%
\begin{frame}[fragile]{Anpassung der Grundhaltung}

\begin{itemize}
\item Nutze die Befehle, um zu beschrieben „was ist“, nicht „wie es aussieht“.
\item Fokussiere Dich auf den Inhalt.
\item Lass \LaTeX{} den Rest machen.
\end{itemize}
\end{frame}

%%%%%%%%%%%%%%%%%%%%%%%%%%%%%%%%%%%%%%%%%%%%%%%%%%%%%%%%%%%%%%%%%%%%%%%%%%%%%%%
%%%%%%%%%%%%%%%%%%%%%%%%%%%%%%%%%%%%%%%%%%%%%%%%%%%%%%%%%%%%%%%%%%%%%%%%%%%%%%%
%%%%%%%%%%%%%%%%%%%%%%%%%%%%%%%%%%%%%%%%%%%%%%%%%%%%%%%%%%%%%%%%%%%%%%%%%%%%%%%
\section{Die Grundlagen}

%%%%%%%%%%%%%%%%%%%%%%%%%%%%%%%%%%%%%%%%%%%%%%%%%%%%%%%%%%%%%%%%%%%%%%%%%%%%%%%
%%%%%%%%%%%%%%%%%%%%%%%%%%%%%%%%%%%%%%%%%%%%%%%%%%%%%%%%%%%%%%%%%%%%%%%%%%%%%%%
%%%%%%%%%%%%%%%%%%%%%%%%%%%%%%%%%%%%%%%%%%%%%%%%%%%%%%%%%%%%%%%%%%%%%%%%%%%%%%%
\subsection{Erste Schritte}
\begin{frame}[fragile]{\insertsubsection}
\begin{itemize}
\item Ein Minimalbeispiel:
\inputminted[frame=single]{latex}{basics.tex}
\item Befehle starten mit einem \emph{Backslash} \keystrokebftt{\bs}.
\item Jedes Dokument beginnt mit einem \cmdbs{documentclass}-Befehl.
\item Ein \emph{Argument} in geschweiften Klammern \keystrokebftt{\{} \keystrokebftt{\}} zeigt \LaTeX{}, 
welche Art von Dokument es erstellen soll: einen \bftt{article}.
\item Ein Prozentzeichen \keystrokebftt{\%} startet einen \emph{Kommentar} --- \LaTeX{}
ignoriert den Rest der Zeile.
\end{itemize}
\end{frame}

%%%%%%%%%%%%%%%%%%%%%%%%%%%%%%%%%%%%%%%%%%%%%%%%%%%%%%%%%%%%%%%%%%%%%%%%%%%%%%%
%%%%%%%%%%%%%%%%%%%%%%%%%%%%%%%%%%%%%%%%%%%%%%%%%%%%%%%%%%%%%%%%%%%%%%%%%%%%%%%
%%%%%%%%%%%%%%%%%%%%%%%%%%%%%%%%%%%%%%%%%%%%%%%%%%%%%%%%%%%%%%%%%%%%%%%%%%%%%%%
\begin{frame}[fragile]{\insertsubsection{} mit \wllogo}
\begin{itemize}
\item Overleaf ist eine Website zur Erstellung von Dokumenten in \LaTeX.
\item Es kompiliert Dein \LaTeX{} automatisch und zeigt Dir das Resultat.
\vskip 2em
\begin{center}
\fbox{\href{\wlnewdoc{basics.tex}}{%
Klicke hier, um ein Beispiel-Dokument in \wllogo{} zu öffnen}}
\\[1ex]\scriptsize{}
\href{http://www.google.com/chrome}{Google Chrome} oder eine neuere Version von \href{http://www.mozilla.org/en-US/firefox/new/}{Firefox} garantieren die besten Ergebnisse.
\end{center}
\vskip 2ex
\item Während wir durch die Folien gehen, probiere die Beispiele in Overleaf aus.
\item \textbf{Wirklich, Du solltest die Befehle selber durchführen!}
\end{itemize}
\end{frame}

%%%%%%%%%%%%%%%%%%%%%%%%%%%%%%%%%%%%%%%%%%%%%%%%%%%%%%%%%%%%%%%%%%%%%%%%%%%%%%%
%%%%%%%%%%%%%%%%%%%%%%%%%%%%%%%%%%%%%%%%%%%%%%%%%%%%%%%%%%%%%%%%%%%%%%%%%%%%%%%
%%%%%%%%%%%%%%%%%%%%%%%%%%%%%%%%%%%%%%%%%%%%%%%%%%%%%%%%%%%%%%%%%%%%%%%%%%%%%%%
\subsection{Textsatz}
\begin{frame}[fragile]{\insertsubsection{}}
\small
\begin{itemize}
\item Tippe Deinen Text zwische \cmdbegin{document} und \cmdend{document}.
\item Meistens kannst Du den Text ganz normal eingeben. 
\begin{exampletwouptiny}
En Wort wird durch ein oder 
mehrere Leerzeichen getrennt.

Ein Absatz wird durch eine oder 
mehrere leere Zeilen 
vom letzten Absatz getrennt.
\end{exampletwouptiny}
\item Abstände in der Quelldatei werden im gesetzten Dokument entfernt.
\begin{exampletwouptiny}
\LaTeX{}   ist           wirklich 
extrem     hilfreich.
\end{exampletwouptiny}
\end{itemize}
\end{frame}

%%%%%%%%%%%%%%%%%%%%%%%%%%%%%%%%%%%%%%%%%%%%%%%%%%%%%%%%%%%%%%%%%%%%%%%%%%%%%%%
%%%%%%%%%%%%%%%%%%%%%%%%%%%%%%%%%%%%%%%%%%%%%%%%%%%%%%%%%%%%%%%%%%%%%%%%%%%%%%%
%%%%%%%%%%%%%%%%%%%%%%%%%%%%%%%%%%%%%%%%%%%%%%%%%%%%%%%%%%%%%%%%%%%%%%%%%%%%%%%
\begin{frame}[fragile]{\insertsubsection{}: Einschränkungen}
\small
\begin{itemize}
\item Einige gebräuchliche Zeichen eine besondere Bedeutung in in \LaTeX:\\[1ex]
\begin{tabular}{cl}
\keystrokebftt{\%} & Prozent           \\
\keystrokebftt{\#} & Raute  \\
\keystrokebftt{\&} & „Und-Zeichen“                 \\
\keystrokebftt{\$} & Dollar              \\
\end{tabular}
\item Wenn Du diese Zeichen benutzt, erscheint eine Fehlermeldung. 
Wenn Du das tatsächliche Symbol nutzen willst, musst Du einen 
\emph{Gegenschrägstrich} (Backslash) vor das Zeichen setzen. 
\begin{exampletwoup}
\$\%\&\#!
\end{exampletwoup}
\end{itemize}
\end{frame}

%%%%%%%%%%%%%%%%%%%%%%%%%%%%%%%%%%%%%%%%%%%%%%%%%%%%%%%%%%%%%%%%%%%%%%%%%%%%%%%
%%%%%%%%%%%%%%%%%%%%%%%%%%%%%%%%%%%%%%%%%%%%%%%%%%%%%%%%%%%%%%%%%%%%%%%%%%%%%%%
%%%%%%%%%%%%%%%%%%%%%%%%%%%%%%%%%%%%%%%%%%%%%%%%%%%%%%%%%%%%%%%%%%%%%%%%%%%%%%%
\begin{frame}[fragile]{Handling Errors}
\begin{itemize}
\item \LaTeX{} can get confused when it is trying to compile your document. If
it does, it stops with an error, which you must fix before it will produce
any output.
\item For example, if you misspell \cmdbs{emph} as \cmdbs{meph}, \LaTeX{} will
stop with an ``undefined control sequence'' error, because ``meph'' is not
one of the commands it knows.
\end{itemize}
\begin{block}{Advice on Errors}
\begin{enumerate}
\item Don't panic! Errors happen.
\item Fix them as soon as they arise --- if what you just typed caused an error,
you can start your debugging there.
\item If there are multiple errors, start with the first one --- the cause may
even be above it.
\end{enumerate}
\end{block}
\end{frame}

%%%%%%%%%%%%%%%%%%%%%%%%%%%%%%%%%%%%%%%%%%%%%%%%%%%%%%%%%%%%%%%%%%%%%%%%%%%%%%%
%%%%%%%%%%%%%%%%%%%%%%%%%%%%%%%%%%%%%%%%%%%%%%%%%%%%%%%%%%%%%%%%%%%%%%%%%%%%%%%
%%%%%%%%%%%%%%%%%%%%%%%%%%%%%%%%%%%%%%%%%%%%%%%%%%%%%%%%%%%%%%%%%%%%%%%%%%%%%%%
\begin{frame}[fragile]{Typesetting Exercise 1}

\begin{block}{Typeset this in \LaTeX:
\footnote{\url{http://en.wikipedia.org/wiki/Economy_of_the_United_States}}}
In March 2006, Congress raised that ceiling an additional \$0.79 trillion to
\$8.97 trillion, which is approximately 68\% of GDP. As of October 4, 2008, the
``Emergency Economic Stabilization Act of 2008'' raised the current debt ceiling
to \$11.3 trillion.
\end{block}
\vskip 2ex
\begin{center}
\fbox{\href{\wlnewdoc{basics-exercise-1.tex}}{%
Click to open this exercise in \wllogo{}}}
\end{center}

\begin{itemize}
\item Hint: watch out for characters with special meanings!
\item Once you've tried,
\fbox{\href{\wlnewdoc{basics-exercise-1-solution.tex}}{%
click here to see my solution}}.
\end{itemize}
\end{frame}

%%%%%%%%%%%%%%%%%%%%%%%%%%%%%%%%%%%%%%%%%%%%%%%%%%%%%%%%%%%%%%%%%%%%%%%%%%%%%%%
%%%%%%%%%%%%%%%%%%%%%%%%%%%%%%%%%%%%%%%%%%%%%%%%%%%%%%%%%%%%%%%%%%%%%%%%%%%%%%%
%%%%%%%%%%%%%%%%%%%%%%%%%%%%%%%%%%%%%%%%%%%%%%%%%%%%%%%%%%%%%%%%%%%%%%%%%%%%%%%
\subsection{Typesetting Mathematics}
\begin{frame}[fragile]{\insertsubsection{}: Dollar Signs}
\begin{itemize}
\item Why are dollar signs \keystrokebftt{\$} special? We use them to mark mathematics in text.\\[1ex]
\begin{exampletwouptiny}
% not so good:
Let a and b be distinct positive
integers, and let c = a - b + 1.

% much better:
Let $a$ and $b$ be distinct positive
integers, and let $c = a - b + 1$.
\end{exampletwouptiny}
\item Always use dollar signs in pairs --- one to begin the mathematics, and one
to end it.
\item \LaTeX{} handles spacing automatically; it ignores your spaces.
\begin{exampletwouptiny}
Let $y=mx+b$ be \ldots

Let $y = m x + b$ be \ldots
\end{exampletwouptiny}
\end{itemize}
\end{frame}

%%%%%%%%%%%%%%%%%%%%%%%%%%%%%%%%%%%%%%%%%%%%%%%%%%%%%%%%%%%%%%%%%%%%%%%%%%%%%%%
%%%%%%%%%%%%%%%%%%%%%%%%%%%%%%%%%%%%%%%%%%%%%%%%%%%%%%%%%%%%%%%%%%%%%%%%%%%%%%%
%%%%%%%%%%%%%%%%%%%%%%%%%%%%%%%%%%%%%%%%%%%%%%%%%%%%%%%%%%%%%%%%%%%%%%%%%%%%%%%
\begin{frame}[fragile]{\insertsubsection{}: Notation}
\begin{itemize}
\item Use caret \keystrokebftt{\^} for superscripts and underscore \keystrokebftt{\_} for subscripts.
\begin{exampletwouptiny}
$y = c_2 x^2 + c_1 x + c_0$
\end{exampletwouptiny}
\vskip 2ex

\item Use curly braces \keystrokebftt{\{} \keystrokebftt{\}} to group
superscripts and subscripts.
\begin{exampletwouptiny}
$F_n = F_n-1 + F_n-2$     % oops!

$F_n = F_{n-1} + F_{n-2}$ % ok!
\end{exampletwouptiny}
\vskip 2ex

\item There are commands for Greek letters and common notation.
\begin{exampletwouptiny}
$\mu = A e^{Q/RT}$

$\Omega = \sum_{k=1}^{n} \omega_k$
\end{exampletwouptiny}
\end{itemize}
\end{frame}

%%%%%%%%%%%%%%%%%%%%%%%%%%%%%%%%%%%%%%%%%%%%%%%%%%%%%%%%%%%%%%%%%%%%%%%%%%%%%%%
%%%%%%%%%%%%%%%%%%%%%%%%%%%%%%%%%%%%%%%%%%%%%%%%%%%%%%%%%%%%%%%%%%%%%%%%%%%%%%%
%%%%%%%%%%%%%%%%%%%%%%%%%%%%%%%%%%%%%%%%%%%%%%%%%%%%%%%%%%%%%%%%%%%%%%%%%%%%%%%
\begin{frame}[fragile]{\insertsubsection{}: Displayed Equations}
\begin{itemize}
\item If it's big and scary, \emph{display} it on its own line using
\cmdbegin{equation} and \cmdend{equation}.\\[2ex]
\begin{exampletwouptiny}
The roots of a quadratic equation
are given by
\begin{equation}
x = \frac{-b \pm \sqrt{b^2 - 4ac}}
         {2a}
\end{equation}
where $a$, $b$ and $c$ are \ldots
\end{exampletwouptiny}
\vskip 1em
{\scriptsize Caution: \LaTeX{} mostly ignores your spaces in mathematics, but it
can't handle blank lines in equations --- don't put blank lines in your
mathematics.}
\end{itemize}
\end{frame}

%%%%%%%%%%%%%%%%%%%%%%%%%%%%%%%%%%%%%%%%%%%%%%%%%%%%%%%%%%%%%%%%%%%%%%%%%%%%%%%
%%%%%%%%%%%%%%%%%%%%%%%%%%%%%%%%%%%%%%%%%%%%%%%%%%%%%%%%%%%%%%%%%%%%%%%%%%%%%%%
%%%%%%%%%%%%%%%%%%%%%%%%%%%%%%%%%%%%%%%%%%%%%%%%%%%%%%%%%%%%%%%%%%%%%%%%%%%%%%%
\begin{frame}[fragile]{Interlude: Environments}
\begin{itemize}
\item \bftt{equation} is an \emph{environment} --- a context.
\item A command can produce different output in different contexts.
\begin{exampletwouptiny}
We can write
$ \Omega = \sum_{k=1}^{n} \omega_k $
in text, or we can write
\begin{equation}
  \Omega = \sum_{k=1}^{n} \omega_k
\end{equation}
to display it.
\end{exampletwouptiny}
\vskip 2ex
\item Note how the $\Sigma$ is bigger in the \bftt{equation} environment, and
how the subscripts and superscripts change position, even though we used the
same commands.
\vskip 1em
{\scriptsize In fact, we could have written \bftt{\$...\$} as
\cmdbegin{math}\bftt{...}\cmdend{math}.}
\end{itemize}
\end{frame}

%%%%%%%%%%%%%%%%%%%%%%%%%%%%%%%%%%%%%%%%%%%%%%%%%%%%%%%%%%%%%%%%%%%%%%%%%%%%%%%
%%%%%%%%%%%%%%%%%%%%%%%%%%%%%%%%%%%%%%%%%%%%%%%%%%%%%%%%%%%%%%%%%%%%%%%%%%%%%%%
%%%%%%%%%%%%%%%%%%%%%%%%%%%%%%%%%%%%%%%%%%%%%%%%%%%%%%%%%%%%%%%%%%%%%%%%%%%%%%%
\begin{frame}[fragile]{Interlude: Environments}
\begin{itemize}
\item The \cmdbs{begin} and \cmdbs{end} commands are used to create many
different environments.
\vskip 2ex

\item The \bftt{itemize} and \bftt{enumerate} environments generate lists.
\begin{exampletwouptiny}
\begin{itemize} % for bullet points
\item Biscuits
\item Tea
\end{itemize}

\begin{enumerate} % for numbers
\item Biscuits
\item Tea
\end{enumerate}
\end{exampletwouptiny}
\end{itemize}
\end{frame}

%%%%%%%%%%%%%%%%%%%%%%%%%%%%%%%%%%%%%%%%%%%%%%%%%%%%%%%%%%%%%%%%%%%%%%%%%%%%%%%
%%%%%%%%%%%%%%%%%%%%%%%%%%%%%%%%%%%%%%%%%%%%%%%%%%%%%%%%%%%%%%%%%%%%%%%%%%%%%%%
%%%%%%%%%%%%%%%%%%%%%%%%%%%%%%%%%%%%%%%%%%%%%%%%%%%%%%%%%%%%%%%%%%%%%%%%%%%%%%%
\begin{frame}[fragile]{Interlude: Packages}

\begin{itemize}
\item All of the commands and environments we've used so far are built into
\LaTeX.

\item \emph{Packages} are libraries of extra commands and environments. There
are thousands of freely available packages.

\item We have to load each of the packages we want to use with a
\cmdbs{usepackage} command in the \emph{preamble}.

\item Example: \bftt{amsmath} from the American Mathematical Society.
\begin{minted}[fontsize=\small,frame=single]{latex}
\documentclass{article}
\usepackage{amsmath} % preamble
\begin{document}
% now we can use commands from amsmath here...
\end{document}
\end{minted}
\end{itemize}
\end{frame}

%%%%%%%%%%%%%%%%%%%%%%%%%%%%%%%%%%%%%%%%%%%%%%%%%%%%%%%%%%%%%%%%%%%%%%%%%%%%%%%
%%%%%%%%%%%%%%%%%%%%%%%%%%%%%%%%%%%%%%%%%%%%%%%%%%%%%%%%%%%%%%%%%%%%%%%%%%%%%%%
%%%%%%%%%%%%%%%%%%%%%%%%%%%%%%%%%%%%%%%%%%%%%%%%%%%%%%%%%%%%%%%%%%%%%%%%%%%%%%%
\begin{frame}[fragile]{\insertsubsection{}: Examples with \bftt{amsmath}}
\begin{itemize}
\item Use \bftt{equation*} (``equation-star'') for unnumbered equations.
\begin{exampletwouptiny}
\begin{equation*}
  \Omega = \sum_{k=1}^{n} \omega_k
\end{equation*}
\end{exampletwouptiny}
\item \LaTeX{} treats adjacent letters as variables multiplied together, which
is not always what you want. \bftt{amsmath} defines commands for many common
mathematical operators.
\begin{exampletwouptiny}
\begin{equation*} % bad!
 min_{x,y} (1-x)^2 + 100(y-x^2)^2
\end{equation*}
\begin{equation*} % good!
\min_{x,y}{(1-x)^2 + 100(y-x^2)^2}
\end{equation*}
\end{exampletwouptiny}
\item You can use \cmdbs{operatorname} for others.
\begin{exampletwouptiny}
\begin{equation*}
\beta_i =
\frac{\operatorname{Cov}(R_i, R_m)}
     {\operatorname{Var}(R_m)}
\end{equation*}
\end{exampletwouptiny}
\end{itemize}
\end{frame}

%%%%%%%%%%%%%%%%%%%%%%%%%%%%%%%%%%%%%%%%%%%%%%%%%%%%%%%%%%%%%%%%%%%%%%%%%%%%%%%
%%%%%%%%%%%%%%%%%%%%%%%%%%%%%%%%%%%%%%%%%%%%%%%%%%%%%%%%%%%%%%%%%%%%%%%%%%%%%%%
%%%%%%%%%%%%%%%%%%%%%%%%%%%%%%%%%%%%%%%%%%%%%%%%%%%%%%%%%%%%%%%%%%%%%%%%%%%%%%%
\begin{frame}[fragile]{\insertsubsection{}: Examples with \bftt{amsmath}}
\begin{itemize}{\small
\item Align a sequence of equations at the equals sign
\begin{align*}
(x+1)^3 &= (x+1)(x+1)(x+1) \\
        &= (x+1)(x^2 + 2x + 1) \\
        &= x^3 + 3x^2 + 3x + 1
\end{align*}
with the \bftt{align*} environment.

% for whatever reason, this doesn't play well with the twoup environment
\begin{minted}[fontsize=\small,frame=single]{latex}
\begin{align*}
(x+1)^3 &= (x+1)(x+1)(x+1) \\
        &= (x+1)(x^2 + 2x + 1) \\
        &= x^3 + 3x^2 + 3x + 1
\end{align*}
\end{minted}
\item An ampersand \keystrokebftt{\&} separates the left column (before the
$=$) from the right column (after the $=$).
\item A double backslash \keystrokebftt{\bs}\keystrokebftt{\bs} starts a new
line.
}\end{itemize}
\end{frame}


%%%%%%%%%%%%%%%%%%%%%%%%%%%%%%%%%%%%%%%%%%%%%%%%%%%%%%%%%%%%%%%%%%%%%%%%%%%%%%%
%%%%%%%%%%%%%%%%%%%%%%%%%%%%%%%%%%%%%%%%%%%%%%%%%%%%%%%%%%%%%%%%%%%%%%%%%%%%%%%
%%%%%%%%%%%%%%%%%%%%%%%%%%%%%%%%%%%%%%%%%%%%%%%%%%%%%%%%%%%%%%%%%%%%%%%%%%%%%%%
\begin{frame}[fragile]{Typesetting Exercise 2}

\begin{block}{Typeset this in \LaTeX:}
Let $X_1, X_2, \ldots, X_n$ be a sequence of independent and identically
distributed random variables with $\operatorname{E}[X_i] = \mu$ and
$\operatorname{Var}[X_i] = \sigma^2 < \infty$, and let
\begin{equation*}
S_n = \frac{1}{n}\sum_{i}^{n} X_i
\end{equation*}
denote their mean. Then as $n$ approaches infinity, the random variables
$\sqrt{n}(S_n - \mu)$ converge in distribution to a normal $N(0, \sigma^2)$.
\end{block}
\vskip 2ex
\begin{center}
\fbox{\href{\wlnewdoc{basics-exercise-2.tex}}{%
Click to open this exercise in \wllogo{}}}
\end{center}
\begin{itemize}
\item Hint: the command for $\infty$ is \cmdbs{infty}.
\item Once you've tried,
\fbox{\href{\wlnewdoc{basics-exercise-2-solution.tex}}{%
click here to see my solution}}.
\end{itemize}
\end{frame}

%%%%%%%%%%%%%%%%%%%%%%%%%%%%%%%%%%%%%%%%%%%%%%%%%%%%%%%%%%%%%%%%%%%%%%%%%%%%%%%
%%%%%%%%%%%%%%%%%%%%%%%%%%%%%%%%%%%%%%%%%%%%%%%%%%%%%%%%%%%%%%%%%%%%%%%%%%%%%%%
%%%%%%%%%%%%%%%%%%%%%%%%%%%%%%%%%%%%%%%%%%%%%%%%%%%%%%%%%%%%%%%%%%%%%%%%%%%%%%%
\begin{frame}{End of Part 1}
\begin{itemize}
\item Congrats! You've already learned how to \ldots
\begin{itemize}
\item Typeset text in \LaTeX.
\item Use lots of different commands.
\item Handle errors when they arise.
\item Typeset some beautiful mathematics.
\item Use several different environments.
\item Load packages.
\end{itemize}
\item That's amazing!
\item In Part 2, we'll see how to use \LaTeX{} to write structured documents
with sections, cross references, figures, tables and bibliographies. See you
then!
\end{itemize}
\end{frame}

\end{document}
